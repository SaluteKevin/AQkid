\section{ข้อมูลของระบบ}

\subsection{หลักการทำงานของระบบและการแก้ปัญหา}

\subsubsection{การลงทะเบียนเรียนคอร์สเรียนว่ายนํ้า}

\begin{enumerate}
    \item ลูกค้าสมัครใช้งานระบบ
    \item แสดงคอร์สเรียนว่ายนํ้าที่พร้อมให้ลงทะเบียนเรียนทั้งหมด
    \item ลูกค้าสามารถเลือกสมัครคอร์สเรียนว่ายนํ้าที่เปิดสอน และ คุณสมบัติของตัวลูกค้าผ่านตามเงิื่อนไขของคอร์สเรียน
    \item ผู้ดูแลระบบตัดสินใจ อนุมัติ หรือ ปฏิเสธ รายการสมัครคอร์สเรียน
\end{enumerate}

\subsubsection{การลงทะเบียนเรียนล้มเหลว และ การขอคืนเงิน}

\begin{enumerate}
    \item ลูกค้าสามารถขอคืนเงินในกรณีที่การลงทะเบียนเรียนล้มเหลว
    \item กรณีล้มเหลว
        \begin{itemize}
            \item ทำการแนบหลักฐานการชำระเงินไม่ทันเวลาที่กำหนด
            \item ทำการชำระเงินด้วยจำนวนที่ไม่ตรงตามเงื่อนไขการชำระเงิน
        \end{itemize}
    \item ผู้ดูแลระบบตัดสินใจ อนุมัติ หรือ ปฏิเสธ รายการขอคืนเงิน
\end{enumerate}

\subsubsection{การรับสมัครอาจารย์ผู้สอน}

\begin{enumerate}
    \item ผู้ดูแลรับสมัครและตรวจสอบ คุณสมบัติอาจารย์ผู้สอน
    \item ผู้ดูแลสมัครใช้งานระบบให้ อาจารย์ผู้สอน
\end{enumerate}

\subsubsection{การสร้างคอร์สเรียน}

\begin{enumerate}
    \item ผู้ดูแลเลือกสร้างคอร์สเรียนตามวันเวลาในระบบ และ กำหนดคุณสมบัติของคอร์สเรียน
    \item ผู้ดูแลเลือกอาจารย์ผู้สอนมารับผิดชอบการสอนของคอร์สเรียน
\end{enumerate}

\subsubsection{การเข้าเรียน}

\begin{enumerate}
    \item การเข้าเรียนในกรณีปกติ
    \begin{enumerate}
        \item นักเรียนเข้าเรียนคาบเรียนว่ายนํ้าตามวันเวลาที่กำหนด
        \item อาจารย์ผู้สอนเช็คชื่อการเข้าเรียนของนักเรียน
    \end{enumerate}
    \item การเข้าเรียนในกรณีพิเศษ
    \begin{enumerate}
        \item การเรียนร่วม
        \begin{enumerate}
            \item มีเหตุผลที่จำเป็นสำหรับการเรียนร่วม\\
                \textbf{เหตุผลที่จำเป็น}\\
                \begin{itemize}
                    \item นักเรียนไม่สามารถเข้าคาบเรียนในวันเวลาที่กำหนด
                    \item อาจารย์ผู้สอนไม่สามารถสอนในวันเวลาที่กำหนด
                \end{itemize}
        \end{enumerate}
        \item การเรียนเสริม
        \begin{enumerate}
            \item มีเหตุผลที่จำเป็นสำหรับการจัดคาบเรียนเสริม\\
            \textbf{เหตุผลที่จำเป็น}\\
            \begin{itemize}
                \item นักเรียนไม่สามารถเข้าคาบเรียน หรือ คาบเรียนอื่นในวันเวลาที่กำหนด
                \item อาจารย์ผู้สอนไม่สามารถสอนในวันเวลาที่กำหนด
            \end{itemize}
            \item ผู้ดูแลสร้างคาบเรียนเสริม และ จัดสรรนักเรียนที่จะเรียนในคาบเรียนเสริม
        \end{enumerate}
    \end{enumerate}
\end{enumerate}

\subsection{ทฤษฎีที่เกี่ยวข้อง}

ในการวิเคราะห์และออกแบบระบบนั้น มีความจำเป็นที่จะต้องใช้หลักความรู้และความเข้าใจจากหลายๆส่วนด้วยกัน ต้องมีการสำรวจปัญหาและความต้องการ รวมถึงความเหมาะสมของระบบที่จะต้องออกแบบให้เข้ากับผู้ใช้ มีการออกแบบอย่างละเอียดและรอบคอบ โดยต้องมีกาเปรียบเทียบกับระบบเดิมที่ผู้ใช้นั้นใช้งานอยู่ในปัจจุบัน เพื่อที่จะสามารถเห็นถึงความแตกต่างและข้อปรับปรุงที่ควรพัฒนาให้ดียิ่งขึ้น

\subsection{การออกแบบระบบการจัดการฐานข้อมูล}

ในการออกแบบระบบจัดการฐานข้อมูลจำเป็นต้องใช้ความรู้ และ การพิจารณาออกแบบระบบอย่างถี่ถ้วน โดยต้องเข้าใจสิ่งที่ต้องออกแบบเป็นอย่างดีไม่ว่าจะเป็นการทำงานของธุรกิจ เพื่อให้การออกแบบเป็นไปได้ไง และ เป็นขั้นตอน โดยการออกแบบระบบฐานข้อมูลนั้นต้องมีการออกแบบดังนี้ ได้แก่

\begin{itemize}
    \item Data Flow Diagram
    \item CRUD Table
    \item ER Diagram
    \item Table Structure (Query Table, Data Dictionary)
\end{itemize}

\subsection{การกำหนดและออกแบบคุณสมบัติของแอปพลิเคชัน}

ในการพัฒนาและออกแบบคุณสมบัติของแอปพลิเคชันจำเป็นจะต้องมีการวิเคราะห์และสำรวจถึงปัญหา ความจำเป็น รวมถึงความต้องการจากผู้ใช้งานระบบ โดยเมื่อมีการวิเคราะห์เสร็จเรียบร้อยแล้วนั้น จสามารถนำมาเขียนเป็นแผนผังต่างๆและตารางต่างๆ เพื่อทำให้มีความเข้าใจและสามารถพัฒนาได้ตรงตามที่กำหนดไว้

\begin{itemize}
    \item Activity Diagram (Business Process)
    \item Use Case Diagram
    \item Use Case Description
    \item Sequence Diagram
    \item Collaboration Diagram
    \item State Diagram
    \item Class Diagram
\end{itemize}

\subsection{เทคโนโลยีที่ใช้พัฒนาระบบ}

\begin{enumerate}
    \item Visual Studio Code
    \item Docker
    \item Laravel web framework
    \begin{enumerate}
        \item php programming language
        \item guzzle HTTP client
        \item jwt-auth
        \item pusher php server
    \end{enumerate}
    \item Nuxt.js web framework
    \begin{enumerate}
        \item Pinia
        \item day.js
        \item fullcalendar
        \item vueuse
        \item vue datepicker
        \item vue clock outside
        \item gsap
        \item pdfeasy
        \item pusher.js
    \end{enumerate}
    \item draw.io
    \item PlantUML
    \item Mermaid Chart
    \item Google Docs
    \item miro.com
\end{enumerate}
